\documentclass[11pt,letterpaper]{article}

\usepackage{tabularx}
\usepackage{booktabs}
\usepackage{caption}
\usepackage[usenames, dvipsnames]{color}
\usepackage{subcaption}
\usepackage{setspace}
\usepackage{hyperref}
\usepackage{multirow}
\usepackage{minted}
\usepackage{mdframed}
% \usepackage{lineno}
% \usepackage{fullpage}
% \linenumbers
\usepackage{pslatex}
\usepackage{apacite}
\usepackage{graphicx}
\usepackage[margin=1.0in]{geometry}
\usepackage{pgfplots}
\usepackage{xcolor, colortbl}
\hypersetup{
    colorlinks,
    linkcolor={cyan!20!red},
    urlcolor={cyan!80!green},
    citecolor={cyan!80!blue}
}

\definecolor{light-gray}{HTML}{F0F0F0}

% \usepackage{tabularx,adjustbox}

\usepackage{gb4e}  % linguistic examples
\noautomath

\usepackage{amsmath}

\newcommand{\pdone}{\tau_1^*}
\newcommand{\pdtwo}{\tau_2^*}
\newcommand{\pds}{(\pd1, \pd2)}
\newcommand{\fg}{f_{ground}}
\newcommand{\fe}{f_{excited}}

\title{Notes on cultural evolution of emergent group-level traits applied to 
  bifurcation of political language and conceptualization \\ {\small COGS 269 Fall 2017 Final Paper}}

\author{{\bf Matthew Turner}}

\begin{document}

\maketitle

\section{Introduction}\label{intro}

This paper explores some issues in language use as a group-level trait. 
I begin with a motivation below: I think a verbal model 
articulated by George Lakoff is promising, but lacks a serious theoretical
social science foundation. I believe a formal model of cultural
group selection that accounted for emergent group formation could provide
the necessary foundation for Lakoff's verbal model. Since Lakoff's model 
is about the distribution of concepts in individuals' heads, it's necessary
to have a representation of concepts, and I suggest one that I think could work
in an emergent group-level trait framework. I continue on to explore the 
iterated learning model for the evolution of language, and evaluate it in 
comparison with a fledgling formal model of group language use as an
emergent group-level trait.

\subsection{Motivation} \label{motivation}

What causes members of a population to interpret the same 
word in different ways? The case of a word having multiple meanings is 
technically known as polysemy. Language has the problem of persistent 
underdeterminism because the world around us is constantly changing. If 
the world to which our language ultimately refers is changing faster than
language, language can't express everything in the world. However, language
changes more rapidly than genes. In this way, languages can fit the mold of
biology, but better language can serve as a basin of attraction for genetic
evolution. In other words, language/communication evolves culturally 
towards optimality given a particular biological machine. However, this 
biological machine encoding will self-optimize for the language family that 
is used among the population it encodes. 
Language is a clumsy way to talk about language, is it not? 

Language also seems
to be a social marker: we identify who to work with based on the language they
speak \cite{Smaldino2017b}. 
The purpose of this review is to identify the linkages between language evolution
and the cultural evolution of groups and group-level traits.

In the political realm, polysemy of words like \emph{justice}, \emph{liberty},
and \emph{freedom} is unavoidable. George Lakoff has suggested 
bifurcation of political opinion and conceptualization in the United States is
due to the
human tendency to think of the nation as a family. He argues ``conservatives''
and ``progressives'' don't share the same conception of the family prototype.
Lakoff claims the progressive family prototype is that of a ``nurturant parent family,''
where parent roles are not gendered and corporal punishment is not used or
accepted. On the other hand, in Lakoff's model, conservatives
conceptual prototype of the family is supposed to be a ``strict father family.''
In the strict father family the father has ultimate and unquestioned 
sovereignty over all other members of the household. Violence is to be used
to protect that sovereignty \cite{Lakoff1996}. 
While the details of this model are questionable
on a number of grounds, the general idea that concepts are interrelated,
interdependent, and hierarchical has about a forty-year history in cognitive
science, starting perhaps with \citeA{Rosch1975}. I am inspired by
Lakoff's work to better understand this conceptual bifurcation, and to 
express his ideas in the language of \textit{The Cultural Evolution of 
Emergent Group-level Traits} as expressed by \cite{Smaldino2014}.

I think Lakoff's hypothesis 
could be addressed and formalized using the appropriate cultural evolutionary theory.
One important question cultural evolutionary theory could answer is,
What is the direction of causality, if there is one, in the case of
conceptual bifurcation in Unites States politics? 
That is, were there first progressives and
conservatives, or were there first differences of family conceptualization, and
in either case how do these concepts interact? 
\citeA{Lakoff2014} provides a framework to 
develop a neurological understanding of conceptual bifurcation, which could
assist in further development of the psychological underpinnings of any 
cultural evolutionary model.

One successful approach to modeling language evolution is the iterated learning
model. One issue is that language evolution dynamics are lost as solutions are
found only for equilibrium distributions of language types \cite{Smith2008}.
Of course in biology a being in equilibrium is dead! Therefore I suggest an 
approach to studying the dynamics of iterated learning that I hope will
have applications to quantifying and modeling the evolution of language away
from equilibrium. New measures for ``polarization'' can define what a group is,
within a certain tolerance. Information theory will help us quantify the difference
between clusters.

\section{The Cultural Evolution of Emergent Group-level Traits}
\label{sec:smaldino-summary}

I will try to argue that language, or, equivalently, internal conceptual relationships,
could be considered an emergent group-level trait, as articulated by
\citeA{Smaldino2014}. An emergent group-level trait is, for example, ``the
music rather than the rock band,'' or ``the sailing ship's voyage rather than
the crew positions.'' Thus, the emergent behavior is the group-level trait. 
This is distinguished from collective behaviors, such as flocking. Collective
behaviors ``result from a number of interchangable individuals acting
independently.'' In another example, Smaldino compares a Roman legion to 
a barbarian horde, suggesting that legioning allows individually-weaker
Roman legionnaires to defeat individually-stronger barbarians. Thus, to map
this to the Lakoff model, I suppose the emergent group-level trait 
for conservatives is ``conserving'' and progressives it's ``progressing.''
In seriousness, though, this model seems to pose some challenges for naming
these not-collective-but-emergent-group-level behaviors. 

Why, in the theory of the cultural evolution of emergent group-level traits
should music or a ship's voyage be the replicator that's ``selected for?'' 
In the emergent group-level trait framework, ``group-level traits
exist fundamentally at the level of groups and can therefore only be defined 
in those terms.'' As a first step toward recasting Lakoff's hypothesis, 
I believe we can identify some combinations of traits that enable
the differentiation between progressive and conservative. In this way, being
progressive is still the emergent group-level trait. Progressive 
political action requires not just a single set of trait values, but the
interaction of many individuals' traits. The challenge, that I don't see
a clear answer to, is how to assign a value to a particular
emergent group-level trait, either in an absolute sense or in relationship
to other emergent group-level traits. 

One challenge I see that I do have an idea how to answer is that of determining
how many groups there are at any given time. A preliminary challenge, for
applying the theory of \citeA{Smaldino2014} to the Lakoff problem, is to
determine a representation for concepts of an individual that can combine with 
other conceptual representations of other individuals to create 
emergent group-level conceptual systems. There are many ways for the meaning
of words to be associated with the meanings of other words. One popular
one is a vector space model of word embeddings in a corpus, for example
using the word2vec algorithm \cite{Mikolov2013a}. Each word appearing in
the corpus is assigned a vector based on how often it appears in the same
phrase with other words. Leaving out many details, one is left with a data
structure in which every row is a $M-$dimensional representation of the
word's meaning. Amazingly, using word2vec, one can perform conceptual 
arithmetic with properly trained models. For example, using the Google News
word embeddings provided as part of the code and data accompaniment to 
\citeA{Mikolov2013a}, the closest vector to the nearest word vector to the 
result of the vector arithmetic \texttt{Germany + France - Paris} is  
\texttt{Berlin}. So, I propose that we can represent a collection of concepts
as a matrix, or equivalently a flattened vector of the entire conceptual system. 
Then, assuming we are working with flattened concept vectors, we can define
a similarity measure between individuals' conceptual system vectors. Given
that, some kind of clustering model could be fit to the
points in conceptual space defined by each individual's conceptual system. 
The distribution within clusters can tell us about how each cluster might
generate group-level traits that persist because they have found a way to
be self-perpetuating. An oustanding question: mustn't this self-perpetuation
still act through increasing individual payoffs?

\section{Iterated learning models of language evolution}

Iterated learning is meant to model the interaction between biological 
biases towards certain types of language and the cultural evolution of
language. 
The iterated learning model (ILM) is a specialized Markov model. In a Markov
model, the future state depends on only the previous state of the system, 
a property known as the Markov property. 
a population is split into $N$ subpopulations,
with the percentage of the $i^{th}$ subpopulation denoted $p_i$. For iterated
learning of language, $p_i$ is said to be the fraction of the population
speaking language $i$. When we want
to include time dependence, we add a $t$ index after a comma, $p_{i,t}$.
The difference equation for the prevalence of each subpopulation is

\begin{equation}
  p_{i,t+1} = \sum_{j=1}^N Q_{ji} p_{j,t}
\label{eq:update}
\end{equation}
\noindent
$Q_{ij}$ is the transition matrix. We could write this more compactly in
matrix-vector notation as $p_{t+1} = Qp_{t}$, where $p_t$ is the vector of all
$p_{i,t}$. In ILM $Q$ is assumed to be time independent;  % possibly remove
when the transition matrix is time independent, the Markov chain is  % possibly remove
labelled \emph{homogeneous} \cite[p. 445]{Griffiths2007a}. The ILM further assumes  % possibly remove
that the transition matrix has exactly one eigenvalue of unit magnitude \cite[p. 446]{Griffiths2007a}.  % possibly remove
Briefly, this means that there exists a \emph{stationary state} 
$p^*$ such that $p^* = Qp^*$. In other words, no iteration needs to be 
done to find the final state of the system, one needs only to find the first
eigenvector of $Q$, and it turns out to be $p^*$. An alternative, intuitive
definition is

\begin{equation}
  p^* = \lim_{t \to \infty} Q^t p_{t=0}
\end{equation}
\noindent
for any valid distribution $p_{t=0}$, i.e. $\sum_i p_{i,t=0} = 1$. 
In other words, we could equivalently find the stationary state if we iterated
as defined in Equation~\ref{eq:update}
until $|p_{t+1} - p_{t}|$ had decreased to below some tolerance.

Up to this point, I have just presented some facts about the type of Markov
chains assumed to be at work in the ILM. We have covered the ``I'' in ILM:
iteration. Where is ``learning''? It is in the values of $Q_{ij}$ themselves.
In the iterated learning model of language change, these represent the 
probability that a speaker of language $i$ at time $t$ will speak language
$j$ at time $t+1$. Each speaker is assumed to be 
exposed to a certain number of utterances, or data, $d$, at each time step. 
The learner in this model must decide which language they heard based on this
data and choose the language that they believe is most representative of the
data they have observed. In terms of probabilities, we modify notation of 
Equation 4.6 in \cite{Smith2008} and write

\begin{equation}
  Q_{ij} = P(\text{choose $i$ at $t+1$}~|~\text{speaker of $j$ at $t$}) = \sum_d P_L(\text{choose $i$}|d) P(d|\text{speaker of $j$})
  \label{eq:qij-elements}
\end{equation}

The goals of ILM are typically to characterize $p^*$ for some set of 
parameters. These parameters include the data ``bottleneck,'' or how many example data $d$ 
a learner sees, i.e.\ the number of summands in Equation~ref{eq:qij-elements}.
Learning is parameterized by choice of calculating these
probabilities written in Equation~\ref{eq:qij-elements}. Further detail about
how these probabilities are calculated is important, but left to the 
references. The important takeaway is that the dynamics are entirely determined
by a time-invariant transition matrix $Q$. The typical approach is to find
a stationary distribution $p^*$ of $Q$, then analyze how the stationary
distribution changes with changes in learning parameters, that eventually
result in changes in the calculation of probabilities in Equation \ref{eq:qij-elements}.

Immediately we can see that this particular approach will not do for
emergent group-level traits, or any \emph{emergent} trait for that matter. 
Any emergence that had occurred happened before the stationary state was
reached. Stationary states are in equilibrium, and equilibrium is death! 
Perhaps Markov chains could be interesting, but it would have to be in 
the time before a stationary state was reached. Transition matrices
could be based on data in the form of expressions of traits. In the
Lakoff-inspired concepts model I proposed, traits are vector representations of 
concepts. Learners could
have a similar updating strategy as in ILM. What we gain is the capability
to simultaneously track the flow of information within and between groups
based on the differences in traits. Given the right dynamics, groups could
come into and out of existence. When equilibrium is reached, the groups that
remain can be identified as attractor groups.

\section{Conclusion}\label{conclusion}

Conceptualizing language use as the exchange of information about individual
traits, we have seen the potential of how iterated learning methods could
work with the framework of emergent group-level traits. Iterated learning
in the limit of $t \to \infty$, as practiced by \cite{Smith2008}, 
will not reveal emergence. One response from Doebeli \& Simon to the target
article of \cite{Smaldino2014} suggested their model \cite{Simon2013} should be adapted and
then adopted as the formal model of cultural evolution of emergent group-level
traits. However, this model they suggest
is not directly applicable, most importantly because its model only allows
parent-child group divergence. An individual may not change groups in its lifetime in
this model. Group size change is determined by how many
offspring of a parent in group $i$ are born as being in group $j$. Clearly 
pre-existing groups that one could be born into are not appropriate for
emergent group-level traits. This claim would require that groups and
the group-level traits define a group, which I think holds.
\citeA{Simon2013} do provide an
important framework for building increasingly expressive models of 
group formation, building a deterministic partial differential equation model
of group selection from a Markovian model \cite{Simon2013}. 
This could be important since trying to fit a mixture model to every 
timestep to identify emergent group-level traits may prove computationally
infeasible, especially as $N$ becomes large. All this is still general, there
is no payoff function yet defined for the individuals that form groups. 
While details must be left until later, I think interesting and relevant
dynamics can be found using a reinforcement learning approach \cite{Sutton2016} 
combined with some kind of economics, so that behavior has a payoff landscape to
navigate, which could result in emergent groups as evidenced by their group-level
traits.


\bibliographystyle{apacite}
\bibliography{/Users/mt/workspace/papers/library.bib}

\end{document}
